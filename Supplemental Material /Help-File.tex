% Options for packages loaded elsewhere
\PassOptionsToPackage{unicode}{hyperref}
\PassOptionsToPackage{hyphens}{url}
%
\documentclass[
]{article}
\usepackage{amsmath,amssymb}
\usepackage{lmodern}
\usepackage{ifxetex,ifluatex}
\ifnum 0\ifxetex 1\fi\ifluatex 1\fi=0 % if pdftex
  \usepackage[T1]{fontenc}
  \usepackage[utf8]{inputenc}
  \usepackage{textcomp} % provide euro and other symbols
\else % if luatex or xetex
  \usepackage{unicode-math}
  \defaultfontfeatures{Scale=MatchLowercase}
  \defaultfontfeatures[\rmfamily]{Ligatures=TeX,Scale=1}
\fi
% Use upquote if available, for straight quotes in verbatim environments
\IfFileExists{upquote.sty}{\usepackage{upquote}}{}
\IfFileExists{microtype.sty}{% use microtype if available
  \usepackage[]{microtype}
  \UseMicrotypeSet[protrusion]{basicmath} % disable protrusion for tt fonts
}{}
\makeatletter
\@ifundefined{KOMAClassName}{% if non-KOMA class
  \IfFileExists{parskip.sty}{%
    \usepackage{parskip}
  }{% else
    \setlength{\parindent}{0pt}
    \setlength{\parskip}{6pt plus 2pt minus 1pt}}
}{% if KOMA class
  \KOMAoptions{parskip=half}}
\makeatother
\usepackage{xcolor}
\IfFileExists{xurl.sty}{\usepackage{xurl}}{} % add URL line breaks if available
\IfFileExists{bookmark.sty}{\usepackage{bookmark}}{\usepackage{hyperref}}
\hypersetup{
  pdftitle={Help file on how to find help},
  pdfauthor={Hannah Massenbauer},
  hidelinks,
  pdfcreator={LaTeX via pandoc}}
\urlstyle{same} % disable monospaced font for URLs
\usepackage[margin=1in]{geometry}
\usepackage{color}
\usepackage{fancyvrb}
\newcommand{\VerbBar}{|}
\newcommand{\VERB}{\Verb[commandchars=\\\{\}]}
\DefineVerbatimEnvironment{Highlighting}{Verbatim}{commandchars=\\\{\}}
% Add ',fontsize=\small' for more characters per line
\usepackage{framed}
\definecolor{shadecolor}{RGB}{248,248,248}
\newenvironment{Shaded}{\begin{snugshade}}{\end{snugshade}}
\newcommand{\AlertTok}[1]{\textcolor[rgb]{0.94,0.16,0.16}{#1}}
\newcommand{\AnnotationTok}[1]{\textcolor[rgb]{0.56,0.35,0.01}{\textbf{\textit{#1}}}}
\newcommand{\AttributeTok}[1]{\textcolor[rgb]{0.77,0.63,0.00}{#1}}
\newcommand{\BaseNTok}[1]{\textcolor[rgb]{0.00,0.00,0.81}{#1}}
\newcommand{\BuiltInTok}[1]{#1}
\newcommand{\CharTok}[1]{\textcolor[rgb]{0.31,0.60,0.02}{#1}}
\newcommand{\CommentTok}[1]{\textcolor[rgb]{0.56,0.35,0.01}{\textit{#1}}}
\newcommand{\CommentVarTok}[1]{\textcolor[rgb]{0.56,0.35,0.01}{\textbf{\textit{#1}}}}
\newcommand{\ConstantTok}[1]{\textcolor[rgb]{0.00,0.00,0.00}{#1}}
\newcommand{\ControlFlowTok}[1]{\textcolor[rgb]{0.13,0.29,0.53}{\textbf{#1}}}
\newcommand{\DataTypeTok}[1]{\textcolor[rgb]{0.13,0.29,0.53}{#1}}
\newcommand{\DecValTok}[1]{\textcolor[rgb]{0.00,0.00,0.81}{#1}}
\newcommand{\DocumentationTok}[1]{\textcolor[rgb]{0.56,0.35,0.01}{\textbf{\textit{#1}}}}
\newcommand{\ErrorTok}[1]{\textcolor[rgb]{0.64,0.00,0.00}{\textbf{#1}}}
\newcommand{\ExtensionTok}[1]{#1}
\newcommand{\FloatTok}[1]{\textcolor[rgb]{0.00,0.00,0.81}{#1}}
\newcommand{\FunctionTok}[1]{\textcolor[rgb]{0.00,0.00,0.00}{#1}}
\newcommand{\ImportTok}[1]{#1}
\newcommand{\InformationTok}[1]{\textcolor[rgb]{0.56,0.35,0.01}{\textbf{\textit{#1}}}}
\newcommand{\KeywordTok}[1]{\textcolor[rgb]{0.13,0.29,0.53}{\textbf{#1}}}
\newcommand{\NormalTok}[1]{#1}
\newcommand{\OperatorTok}[1]{\textcolor[rgb]{0.81,0.36,0.00}{\textbf{#1}}}
\newcommand{\OtherTok}[1]{\textcolor[rgb]{0.56,0.35,0.01}{#1}}
\newcommand{\PreprocessorTok}[1]{\textcolor[rgb]{0.56,0.35,0.01}{\textit{#1}}}
\newcommand{\RegionMarkerTok}[1]{#1}
\newcommand{\SpecialCharTok}[1]{\textcolor[rgb]{0.00,0.00,0.00}{#1}}
\newcommand{\SpecialStringTok}[1]{\textcolor[rgb]{0.31,0.60,0.02}{#1}}
\newcommand{\StringTok}[1]{\textcolor[rgb]{0.31,0.60,0.02}{#1}}
\newcommand{\VariableTok}[1]{\textcolor[rgb]{0.00,0.00,0.00}{#1}}
\newcommand{\VerbatimStringTok}[1]{\textcolor[rgb]{0.31,0.60,0.02}{#1}}
\newcommand{\WarningTok}[1]{\textcolor[rgb]{0.56,0.35,0.01}{\textbf{\textit{#1}}}}
\usepackage{graphicx}
\makeatletter
\def\maxwidth{\ifdim\Gin@nat@width>\linewidth\linewidth\else\Gin@nat@width\fi}
\def\maxheight{\ifdim\Gin@nat@height>\textheight\textheight\else\Gin@nat@height\fi}
\makeatother
% Scale images if necessary, so that they will not overflow the page
% margins by default, and it is still possible to overwrite the defaults
% using explicit options in \includegraphics[width, height, ...]{}
\setkeys{Gin}{width=\maxwidth,height=\maxheight,keepaspectratio}
% Set default figure placement to htbp
\makeatletter
\def\fps@figure{htbp}
\makeatother
\setlength{\emergencystretch}{3em} % prevent overfull lines
\providecommand{\tightlist}{%
  \setlength{\itemsep}{0pt}\setlength{\parskip}{0pt}}
\setcounter{secnumdepth}{-\maxdimen} % remove section numbering
\ifluatex
  \usepackage{selnolig}  % disable illegal ligatures
\fi

\title{Help file on how to find help}
\author{Hannah Massenbauer}
\date{24 2 2024}

\begin{document}
\maketitle

\hypertarget{how-to-find-help}{%
\subsection{How to find help}\label{how-to-find-help}}

When I started working with R I received all the time errors and nothing
worked. This was quit frustrating and after some years and experience I
tried R again and suddenly things worked out. I want to give you an
overview what helps me to deal with errors and other problems, which
occur when working with R.

\hypertarget{tip-use-chatgpt}{%
\subsubsection{1. Tip: Use ChatGPT}\label{tip-use-chatgpt}}

Especially, in the beginning it can be super helpful to get to know the
most important commands and to get a feeling how R works. But, be aware,
how you pose your questions can have a great impact on the answer you
receive. Also ChatGPT will not replace your duty of thinking about what
you want / have to do. Sometimes it errs, but based on the error in the
console window you can copy + paste that back in ChatGPT and in many
cases ChatGPT is able to fix these errors. For basic problems, which I
faced especially in the beginning it is definitely the most helpful tool
to solve problems.

\hypertarget{tip-help-within-r}{%
\subsubsection{2. Tip: Help within R}\label{tip-help-within-r}}

For many commands you can find help within R by typing ``?'' in front of
the command.

\begin{Shaded}
\begin{Highlighting}[]
\NormalTok{?print}
\end{Highlighting}
\end{Shaded}

\begin{verbatim}
## starte den http Server für die Hilfe fertig
\end{verbatim}

\begin{Shaded}
\begin{Highlighting}[]
\NormalTok{?min}
\end{Highlighting}
\end{Shaded}

This is super simple and helps especially with syntax problems. Syntax
problems could imply missing on a comma or having a wrong order in the
command, which results in an error message.

This approach is also recommendable if you want to figure out, what the
command can do and what not. In R there are 1000 ways (that's maybe
exaggerated) to achieve a solution, and not every command will always be
able to do exactly what you need. Thus, check the possibilities of your
command in the help file.

\hypertarget{tip-google}{%
\subsubsection{3. Tip: Google}\label{tip-google}}

\hypertarget{general-questions}{%
\subparagraph{General Questions:}\label{general-questions}}

Many people before you have started learning R. Therefore, most of the
questions you have, were likely posed before by others as well. There
exist some forums, in which you can pose questions and receive answers
from other R-users. ``Stackoverflow''
(\url{https://stackoverflow.com/questions/tagged/r}) is the most used
forum for R questions, but there exist other websites as well. A brief
guide on stackoverflow + other resources you can find here:
\url{https://www.rforecology.com/post/where-to-ask-for-help-when-coding-in-r/}.

The website \url{https://cran.r-project.org/} contains an overview of
the newest packages and commands and especially handy are the manuals
they offer. At the beginning it might look confusing, but when you
google for help you will stumble across their pages and find something
like this:
\url{https://cran.r-project.org/web/packages/Hmisc/index.html} . To find
the help file you click on the ``Reference manual'', which opens a pdf
document with detailed information on packages.

Maybe a bit more advanced place to search for help is Github
(\url{https://github.com/}). It is useful for coding since you can
collaborate with others on the same code and upload your codes such that
you have access to all your codes also in the future. You can also
search for commands or packages within the platform and most users will
provide an explanation file at the same place.

Depending on your interest, or whatever you have to do for a homework
you can google also for specific topics. For example, I like
econometrics and to discover what R codes exist I google ``R
econometrics'' and I discovered this amazing page: Econometrics related:
\url{https://www.econometrics-with-r.org/1-introduction.html}. Another
website I can recommed is ``Medium''. I found the following
\url{https://medium.com/@josef.waples/causal-inference-in-r-using-mtcars-47cb167f9432}
link as an example. One thing I want to point out for all these
unofficial online resources is, however, that there is no guarantee that
they are 100\% correct and it can happen that there is a mistake. So
remain aware, about what you see and what you want to do.

To sum up, there exist numerous sources and with time passing you will
see which pages and formats suits your learning progress the best. When
you don't understand something it depends probably on the way it is
explained and not on you.

\hypertarget{specific-package-questions}{%
\subparagraph{Specific Package
Questions:}\label{specific-package-questions}}

When your R journey continuous, you will discover new packages. At the
beginning it can be confusing which package can do what. To learn what a
specific package is capable of, I would recommend google the package and
you will most likely find a nice website with an explanation of the
package.

Below I provide you an overview of the most important packages + a link
to their website so you can find easily help during this course or also
in your future career:

The probably most important overview for packages are the tidyverse
packages (\url{https://www.tidyverse.org/packages/}) which comprises
multiple packages such as:

\begin{enumerate}
\def\labelenumi{\arabic{enumi}.}
\item
  Readr: Importing data (e.g.~csv)

  \url{https://readr.tidyverse.org/}
\item
  Dplyr : Data wrengling

  \url{https://dplyr.tidyverse.org/}
\item
  Ggplot: Best graph command

  \url{https://ggplot2.tidyverse.org/reference/ggplot.html}
\item
  Tibble: Dataset format

  \url{https://tibble.tidyverse.org/}
\end{enumerate}

But there exist other packages as well like:

\begin{enumerate}
\def\labelenumi{\arabic{enumi}.}
\item
  Data.table:
  \url{https://cran.r-project.org/web/packages/data.table/vignettes/datatable-intro.html}
  (very fast way to work with data)
\item
  Overview:
  \href{https://support.posit.co/hc/en-us/articles/201057987-Quick-list-of-useful-R-packages}{https://support.posit.co/hc/en-us/articles/201057987-Quick-list-of-useful-R-package}
\end{enumerate}

\hypertarget{tip-dont-make-easy-things-hard}{%
\subsubsection{4. Tip: Don't make easy things
hard}\label{tip-dont-make-easy-things-hard}}

Always try to make things as easy as possible. What I mean by that, is
that independent how hard your problem is, try to break it down. Before
attempting to analyze data or execute calculations start by thinking
what is the most straight forward way you can do it, if you can imagine
it, then you can do it! The biggest problem is if you don't know what
you want to do exactly, because searching for solutions will be way
easier if you know what your problem is.

Another tip which makes things easier is to reuse code. If you have one
command / package with which you are familiar, just reuse it in the
future.

Also if you know that a friend has dealt with similar problems, ask them
and just swap codes. You don't have to do everything on your own :)

\end{document}
